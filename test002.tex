\documentclass[titlepage]{jsarticle}
\usepackage{graphicx}



\title{Rough Latex}

\author{kame}

\date{May 18, 2023}


\begin{document}
\maketitle

\begin{abstract}川口氏の案をちょっと試してみる\end{abstract}
<section>
<headline>桃の発見</headline>
ある日、数学者である村人が数式の中で桃が成長する様子を予測する論文を発表しました。その中で彼は次のような数式を提案しました。

桃の成長量 (P) = 初期サイズ (S) * 成長率 (R) ^ 日数 (D)
P=S\timesR^D
彼の予測によれば、初期サイズが小さく成長率が高ければ桃は急速に成長し、初期サイズが大きく成長率が低ければゆっくりと成長するとされていました。

そして、ある日、桃太郎という少年が数学者の論文を読んでいる最中、数式の中で桃が巨大化する様子を想像しました。彼は興味津々でその成長を追いかけるため、実際に桃を栽培することにしました。

<section>
<headline>桃の成長</headline>
桃太郎は数学者の論文に基づいて、桃の成長を計算することにしました。彼は初期サイズを2cm (S = 2)、成長率を1.5倍 (R = 1.5) と仮定しました。そして、日数を増やすごとに桃の成長量を計算しました。

桃の成長量 f(x,y)/x  = 2sinx * 2/(3+2x) ^ {日数 (D)}
\frac{f(x,y)}{x}=2 \sin x\times\frac{2}{(3\plus2x)^D}

彼は日々桃の成長量を計算し、成長の速さに驚きました。1日目には2cmの桃がありますが、2日目には3cm、3日目には4.5cmと急速に成長していくことが分かりました。


\[
\left\{
\begin{array}{ll}
 \sin x\plusy=1 \\
x\plus2y=3 & (c=0,1,2,\cdots)  \\
3x\plusy=\frac{4}{5}\plus10 \\
\end{array}
\right.
\]
\end{document}
